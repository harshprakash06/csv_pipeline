\documentclass[a4paper,11pt]{article}
\usepackage{graphicx}
\usepackage{listings}
\usepackage{hyperref}
\usepackage{geometry}
\geometry{margin=1in}
\usepackage{xcolor}
\usepackage{titlesec}
\usepackage{enumitem}
\usepackage{fancyhdr}
\usepackage{amsmath}
\usepackage{tcolorbox}
\usepackage{mdframed}

% Enhanced color scheme
\definecolor{codeblue}{RGB}{40,116,166}
\definecolor{codegray}{RGB}{128,128,128}
\definecolor{codegreen}{RGB}{34,139,34}
\definecolor{backcolour}{RGB}{248,248,248}
\definecolor{sectioncolor}{RGB}{25,25,112}
\definecolor{headercolor}{RGB}{70,130,180}

% Page setup
\pagestyle{fancy}
\fancyhf{}
\fancyhead[L]{\textcolor{headercolor}{\textbf{Data Pipeline Documentation}}}
\fancyhead[R]{\textcolor{headercolor}{\today}}
\fancyfoot[C]{\textcolor{headercolor}{\thepage}}
\renewcommand{\headrulewidth}{2pt}
\renewcommand{\footrulewidth}{1pt}
\renewcommand{\headrule}{\hbox to\headwidth{\color{headercolor}\leaders\hrule height \headrulewidth\hfill}}
\renewcommand{\footrule}{\hbox to\headwidth{\color{headercolor}\leaders\hrule height \footrulewidth\hfill}}

\setlength{\parskip}{1em}

% Enhanced section formatting
\titleformat{\section}[block]{\Large\bfseries\color{sectioncolor}}{\thesection}{1em}{}
\titleformat{\subsection}[block]{\large\bfseries\color{sectioncolor}}{\thesubsection}{1em}{}

% Code listing style
\lstdefinestyle{mystyle}{
    backgroundcolor=\color{backcolour},   
    commentstyle=\color{codegreen},
    keywordstyle=\color{codeblue},
    numberstyle=\tiny\color{codegray},
    stringstyle=\color{red},
    basicstyle=\ttfamily\footnotesize,
    breakatwhitespace=false,         
    breaklines=true,                 
    captionpos=b,                    
    keepspaces=true,                 
    numbers=left,                    
    numbersep=5pt,                  
    showspaces=false,                
    showstringspaces=false,
    showtabs=false,                  
    tabsize=2,
    frame=single,
    rulecolor=\color{codeblue}
}
\lstset{style=mystyle}

% Title page setup
\title{
    \vspace{-1in}
    \begin{tcolorbox}[colback=headercolor!10, colframe=headercolor, boxrule=2pt, arc=10pt]
        \centering
        \Huge\textbf{\textcolor{sectioncolor}{DuckDB-based CSV Search Pipeline}}\\
        \vspace{0.5em}
        \Large\textcolor{headercolor}{High-Performance Data Processing Documentation}
    \end{tcolorbox}
}



\begin{document}
\maketitle
\begin{center}
\textcolor{headercolor}{\rule{\linewidth}{2pt}}
\end{center}

\section{Overview}

\begin{tcolorbox}[colback=blue!5, colframe=blue!40!black, title=Project Summary]
This project is a scalable and efficient data processing pipeline built in Python, designed to:
\begin{itemize}[leftmargin=1em]
  \item Convert large CSV files to optimized Parquet format
  \item Load and index the data into DuckDB
  \item Perform high-speed searches using indexed fields
  \item Export filtered results in multiple formats
\end{itemize}
\end{tcolorbox}

\section{Dependencies}

\begin{mdframed}[backgroundcolor=backcolour, linecolor=codeblue, linewidth=2pt]
\textbf{Required Dependencies:}
\begin{itemize}[leftmargin=1em]
  \item Python 3.8+
  \item \texttt{duckdb}
  \item \texttt{pandas}
  \item \texttt{pyarrow}
  \item \texttt{tabulate}
\end{itemize}
\end{mdframed}

\textbf{Installation:}
\begin{lstlisting}[language=bash, caption=Environment Setup]
python -m venv .venv
source .venv/bin/activate
pip install -r requirements.txt
\end{lstlisting}

\section{Pipeline Steps}

\subsection{1. CSV to Parquet Conversion}

\textbf{Script:} \texttt{convert\_to\_parquet.py}

\begin{tcolorbox}[colback=green!5, colframe=green!40!black]
\textbf{Features:}
\begin{itemize}[leftmargin=1em]
  \item Converts large CSV files into Parquet format using chunked streaming.
  \item Automatically creates directories and logs the process.
\end{itemize}
\end{tcolorbox}

\textbf{Usage:}
\begin{lstlisting}[language=bash, caption=CSV to Parquet Conversion]
python convert_to_parquet.py --csv data/sample.csv \
                              --parquet data/sample.parquet \
                              --chunk 1000000 \
                              --load yes
\end{lstlisting}

\subsection{2. Data Loading and Indexing}

\textbf{Script:} \texttt{load\_and\_index.py}

\begin{tcolorbox}[colback=orange!5, colframe=orange!40!black]
\textbf{Functionality:}
\begin{itemize}[leftmargin=1em]
  \item Loads Parquet file into DuckDB.
  \item Creates a persistent table and optionally recreates it.
\end{itemize}
\end{tcolorbox}

\textbf{Note:} No CLI arguments are required for this script.

\subsection{3. Search and Export}

\textbf{Script:} \texttt{search\_and\_export.py}

\begin{tcolorbox}[colback=purple!5, colframe=purple!40!black]
\textbf{Capabilities:}
\begin{itemize}[leftmargin=1em]
  \item Searches indexed or filterable fields using range or equality conditions.
  \item Exports results in CSV, JSON, or Parquet.
\end{itemize}
\end{tcolorbox}

\textbf{Usage Examples:}
\begin{lstlisting}[language=bash, caption=Equality Match Search]
# Equality match
python search_and_export.py --equals col_2=foo col_3=bar \
                            --columns col_2 col_3 col_4 \
                            --format csv
\end{lstlisting}

\begin{lstlisting}[language=bash, caption=Range Filter Search]
# Range filter
python search_and_export.py --range col_0 10 50 \
                            --columns col_0 col_1 \
                            --format json
\end{lstlisting}

\subsection{4. Logging}

\begin{tcolorbox}[colback=gray!10, colframe=gray!50!black, title=Logging System]
Each script writes a timestamped log file under the \texttt{logs/} directory, and also streams logs to the terminal.
\end{tcolorbox}

\section{Error Handling}

\begin{mdframed}[backgroundcolor=red!5, linecolor=red!40!black, linewidth=2pt]
The pipeline includes validations and exception handling for:
\begin{itemize}[leftmargin=1em]
  \item Missing or invalid files
  \item Invalid search fields or columns
  \item DuckDB connection issues
  \item Empty query results
\end{itemize}

All errors are logged in the \texttt{logs/} directory for audit and debugging purposes.
\end{mdframed}

\section{Logging Format}

\begin{tcolorbox}[colback=blue!5, colframe=blue!40!black, title=Log Structure]
Each run produces a detailed log file that contains:
\begin{itemize}[leftmargin=1em]
  \item Timestamp of execution
  \item CLI arguments passed
  \item Number of rows processed
  \item Any warnings or errors encountered
\end{itemize}
\end{tcolorbox}

\textbf{Sample Log Output:}
\begin{lstlisting}[caption=Example Log Output]
2025-06-23 14:30:12 [INFO] Running query: SELECT col_1 FROM data WHERE col_0 BETWEEN 10 AND 50
2025-06-23 14:30:12 [INFO] Found 3,920 rows.
2025-06-23 14:30:12 [INFO] Exported to data/search_output.csv
\end{lstlisting}

\section{Folder Structure}

\begin{lstlisting}[caption=Project Directory Structure]
.
├── data/
│   ├── sample.csv
│   ├── sample.parquet
│   └── engine.duckdb
├── logs/
│   └── *.log
├── convert_to_parquet.py
├── load_and_index.py
├── search_and_export.py
└── requirements.txt
\end{lstlisting}

\section{Appendix: Sample Schema}

\begin{tcolorbox}[colback=yellow!10, colframe=yellow!50!black, title=Dataset Schema]
The test dataset contains 50 columns named \texttt{col\_0} to \texttt{col\_49}, with a mix of types:

\begin{itemize}[leftmargin=1em]
  \item \texttt{col\_0, col\_1, col\_2} are indexed (INTEGER)
  \item \texttt{col\_3} to \texttt{col\_20} are FLOAT
  \item \texttt{col\_21} to \texttt{col\_45} are VARCHAR
  \item \texttt{col\_46} to \texttt{col\_49} may include TEXT/BLOB data
\end{itemize}
\end{tcolorbox}

\vspace{2em}
\begin{center}
\textcolor{headercolor}{\rule{\linewidth}{2pt}}
\end{center}

\section*{Author Information}
\begin{center}
\begin{tcolorbox}[colback=headercolor!10, colframe=headercolor, width=0.6\textwidth]
\centering
\textbf{\Large Harsh Prakash} \\
\vspace{0.5em}
Email: \texttt{harshprakash06@gmail.com}\newline
\date{\today}
\end{tcolorbox}
\end{center}

\end{document}